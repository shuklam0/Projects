\documentclass[11pt]{article}

    \usepackage[breakable]{tcolorbox}
    \usepackage{parskip} % Stop auto-indenting (to mimic markdown behaviour)
    
    \usepackage{iftex}
    \ifPDFTeX
    	\usepackage[T1]{fontenc}
    	\usepackage{mathpazo}
    \else
    	\usepackage{fontspec}
    \fi

    % Basic figure setup, for now with no caption control since it's done
    % automatically by Pandoc (which extracts ![](path) syntax from Markdown).
    \usepackage{graphicx}
    % Maintain compatibility with old templates. Remove in nbconvert 6.0
    \let\Oldincludegraphics\includegraphics
    % Ensure that by default, figures have no caption (until we provide a
    % proper Figure object with a Caption API and a way to capture that
    % in the conversion process - todo).
    \usepackage{caption}
    \DeclareCaptionFormat{nocaption}{}
    \captionsetup{format=nocaption,aboveskip=0pt,belowskip=0pt}

    \usepackage{float}
    \floatplacement{figure}{H} % forces figures to be placed at the correct location
    \usepackage{xcolor} % Allow colors to be defined
    \usepackage{enumerate} % Needed for markdown enumerations to work
    \usepackage{geometry} % Used to adjust the document margins
    \usepackage{amsmath} % Equations
    \usepackage{amssymb} % Equations
    \usepackage{textcomp} % defines textquotesingle
    % Hack from http://tex.stackexchange.com/a/47451/13684:
    \AtBeginDocument{%
        \def\PYZsq{\textquotesingle}% Upright quotes in Pygmentized code
    }
    \usepackage{upquote} % Upright quotes for verbatim code
    \usepackage{eurosym} % defines \euro
    \usepackage[mathletters]{ucs} % Extended unicode (utf-8) support
    \usepackage{fancyvrb} % verbatim replacement that allows latex
    \usepackage{grffile} % extends the file name processing of package graphics 
                         % to support a larger range
    \makeatletter % fix for old versions of grffile with XeLaTeX
    \@ifpackagelater{grffile}{2019/11/01}
    {
      % Do nothing on new versions
    }
    {
      \def\Gread@@xetex#1{%
        \IfFileExists{"\Gin@base".bb}%
        {\Gread@eps{\Gin@base.bb}}%
        {\Gread@@xetex@aux#1}%
      }
    }
    \makeatother
    \usepackage[Export]{adjustbox} % Used to constrain images to a maximum size
    \adjustboxset{max size={0.9\linewidth}{0.9\paperheight}}

    % The hyperref package gives us a pdf with properly built
    % internal navigation ('pdf bookmarks' for the table of contents,
    % internal cross-reference links, web links for URLs, etc.)
    \usepackage{hyperref}
    % The default LaTeX title has an obnoxious amount of whitespace. By default,
    % titling removes some of it. It also provides customization options.
    \usepackage{titling}
    \usepackage{longtable} % longtable support required by pandoc >1.10
    \usepackage{booktabs}  % table support for pandoc > 1.12.2
    \usepackage[inline]{enumitem} % IRkernel/repr support (it uses the enumerate* environment)
    \usepackage[normalem]{ulem} % ulem is needed to support strikethroughs (\sout)
                                % normalem makes italics be italics, not underlines
    \usepackage{mathrsfs}
    

    
    % Colors for the hyperref package
    \definecolor{urlcolor}{rgb}{0,.145,.698}
    \definecolor{linkcolor}{rgb}{.71,0.21,0.01}
    \definecolor{citecolor}{rgb}{.12,.54,.11}

    % ANSI colors
    \definecolor{ansi-black}{HTML}{3E424D}
    \definecolor{ansi-black-intense}{HTML}{282C36}
    \definecolor{ansi-red}{HTML}{E75C58}
    \definecolor{ansi-red-intense}{HTML}{B22B31}
    \definecolor{ansi-green}{HTML}{00A250}
    \definecolor{ansi-green-intense}{HTML}{007427}
    \definecolor{ansi-yellow}{HTML}{DDB62B}
    \definecolor{ansi-yellow-intense}{HTML}{B27D12}
    \definecolor{ansi-blue}{HTML}{208FFB}
    \definecolor{ansi-blue-intense}{HTML}{0065CA}
    \definecolor{ansi-magenta}{HTML}{D160C4}
    \definecolor{ansi-magenta-intense}{HTML}{A03196}
    \definecolor{ansi-cyan}{HTML}{60C6C8}
    \definecolor{ansi-cyan-intense}{HTML}{258F8F}
    \definecolor{ansi-white}{HTML}{C5C1B4}
    \definecolor{ansi-white-intense}{HTML}{A1A6B2}
    \definecolor{ansi-default-inverse-fg}{HTML}{FFFFFF}
    \definecolor{ansi-default-inverse-bg}{HTML}{000000}

    % common color for the border for error outputs.
    \definecolor{outerrorbackground}{HTML}{FFDFDF}

    % commands and environments needed by pandoc snippets
    % extracted from the output of `pandoc -s`
    \providecommand{\tightlist}{%
      \setlength{\itemsep}{0pt}\setlength{\parskip}{0pt}}
    \DefineVerbatimEnvironment{Highlighting}{Verbatim}{commandchars=\\\{\}}
    % Add ',fontsize=\small' for more characters per line
    \newenvironment{Shaded}{}{}
    \newcommand{\KeywordTok}[1]{\textcolor[rgb]{0.00,0.44,0.13}{\textbf{{#1}}}}
    \newcommand{\DataTypeTok}[1]{\textcolor[rgb]{0.56,0.13,0.00}{{#1}}}
    \newcommand{\DecValTok}[1]{\textcolor[rgb]{0.25,0.63,0.44}{{#1}}}
    \newcommand{\BaseNTok}[1]{\textcolor[rgb]{0.25,0.63,0.44}{{#1}}}
    \newcommand{\FloatTok}[1]{\textcolor[rgb]{0.25,0.63,0.44}{{#1}}}
    \newcommand{\CharTok}[1]{\textcolor[rgb]{0.25,0.44,0.63}{{#1}}}
    \newcommand{\StringTok}[1]{\textcolor[rgb]{0.25,0.44,0.63}{{#1}}}
    \newcommand{\CommentTok}[1]{\textcolor[rgb]{0.38,0.63,0.69}{\textit{{#1}}}}
    \newcommand{\OtherTok}[1]{\textcolor[rgb]{0.00,0.44,0.13}{{#1}}}
    \newcommand{\AlertTok}[1]{\textcolor[rgb]{1.00,0.00,0.00}{\textbf{{#1}}}}
    \newcommand{\FunctionTok}[1]{\textcolor[rgb]{0.02,0.16,0.49}{{#1}}}
    \newcommand{\RegionMarkerTok}[1]{{#1}}
    \newcommand{\ErrorTok}[1]{\textcolor[rgb]{1.00,0.00,0.00}{\textbf{{#1}}}}
    \newcommand{\NormalTok}[1]{{#1}}
    
    % Additional commands for more recent versions of Pandoc
    \newcommand{\ConstantTok}[1]{\textcolor[rgb]{0.53,0.00,0.00}{{#1}}}
    \newcommand{\SpecialCharTok}[1]{\textcolor[rgb]{0.25,0.44,0.63}{{#1}}}
    \newcommand{\VerbatimStringTok}[1]{\textcolor[rgb]{0.25,0.44,0.63}{{#1}}}
    \newcommand{\SpecialStringTok}[1]{\textcolor[rgb]{0.73,0.40,0.53}{{#1}}}
    \newcommand{\ImportTok}[1]{{#1}}
    \newcommand{\DocumentationTok}[1]{\textcolor[rgb]{0.73,0.13,0.13}{\textit{{#1}}}}
    \newcommand{\AnnotationTok}[1]{\textcolor[rgb]{0.38,0.63,0.69}{\textbf{\textit{{#1}}}}}
    \newcommand{\CommentVarTok}[1]{\textcolor[rgb]{0.38,0.63,0.69}{\textbf{\textit{{#1}}}}}
    \newcommand{\VariableTok}[1]{\textcolor[rgb]{0.10,0.09,0.49}{{#1}}}
    \newcommand{\ControlFlowTok}[1]{\textcolor[rgb]{0.00,0.44,0.13}{\textbf{{#1}}}}
    \newcommand{\OperatorTok}[1]{\textcolor[rgb]{0.40,0.40,0.40}{{#1}}}
    \newcommand{\BuiltInTok}[1]{{#1}}
    \newcommand{\ExtensionTok}[1]{{#1}}
    \newcommand{\PreprocessorTok}[1]{\textcolor[rgb]{0.74,0.48,0.00}{{#1}}}
    \newcommand{\AttributeTok}[1]{\textcolor[rgb]{0.49,0.56,0.16}{{#1}}}
    \newcommand{\InformationTok}[1]{\textcolor[rgb]{0.38,0.63,0.69}{\textbf{\textit{{#1}}}}}
    \newcommand{\WarningTok}[1]{\textcolor[rgb]{0.38,0.63,0.69}{\textbf{\textit{{#1}}}}}
    
    
    % Define a nice break command that doesn't care if a line doesn't already
    % exist.
    \def\br{\hspace*{\fill} \\* }
    % Math Jax compatibility definitions
    \def\gt{>}
    \def\lt{<}
    \let\Oldtex\TeX
    \let\Oldlatex\LaTeX
    \renewcommand{\TeX}{\textrm{\Oldtex}}
    \renewcommand{\LaTeX}{\textrm{\Oldlatex}}
    % Document parameters
    % Document title
    \title{Unison\_Markdown}
    
    
    
    
    
% Pygments definitions
\makeatletter
\def\PY@reset{\let\PY@it=\relax \let\PY@bf=\relax%
    \let\PY@ul=\relax \let\PY@tc=\relax%
    \let\PY@bc=\relax \let\PY@ff=\relax}
\def\PY@tok#1{\csname PY@tok@#1\endcsname}
\def\PY@toks#1+{\ifx\relax#1\empty\else%
    \PY@tok{#1}\expandafter\PY@toks\fi}
\def\PY@do#1{\PY@bc{\PY@tc{\PY@ul{%
    \PY@it{\PY@bf{\PY@ff{#1}}}}}}}
\def\PY#1#2{\PY@reset\PY@toks#1+\relax+\PY@do{#2}}

\expandafter\def\csname PY@tok@w\endcsname{\def\PY@tc##1{\textcolor[rgb]{0.73,0.73,0.73}{##1}}}
\expandafter\def\csname PY@tok@c\endcsname{\let\PY@it=\textit\def\PY@tc##1{\textcolor[rgb]{0.25,0.50,0.50}{##1}}}
\expandafter\def\csname PY@tok@cp\endcsname{\def\PY@tc##1{\textcolor[rgb]{0.74,0.48,0.00}{##1}}}
\expandafter\def\csname PY@tok@k\endcsname{\let\PY@bf=\textbf\def\PY@tc##1{\textcolor[rgb]{0.00,0.50,0.00}{##1}}}
\expandafter\def\csname PY@tok@kp\endcsname{\def\PY@tc##1{\textcolor[rgb]{0.00,0.50,0.00}{##1}}}
\expandafter\def\csname PY@tok@kt\endcsname{\def\PY@tc##1{\textcolor[rgb]{0.69,0.00,0.25}{##1}}}
\expandafter\def\csname PY@tok@o\endcsname{\def\PY@tc##1{\textcolor[rgb]{0.40,0.40,0.40}{##1}}}
\expandafter\def\csname PY@tok@ow\endcsname{\let\PY@bf=\textbf\def\PY@tc##1{\textcolor[rgb]{0.67,0.13,1.00}{##1}}}
\expandafter\def\csname PY@tok@nb\endcsname{\def\PY@tc##1{\textcolor[rgb]{0.00,0.50,0.00}{##1}}}
\expandafter\def\csname PY@tok@nf\endcsname{\def\PY@tc##1{\textcolor[rgb]{0.00,0.00,1.00}{##1}}}
\expandafter\def\csname PY@tok@nc\endcsname{\let\PY@bf=\textbf\def\PY@tc##1{\textcolor[rgb]{0.00,0.00,1.00}{##1}}}
\expandafter\def\csname PY@tok@nn\endcsname{\let\PY@bf=\textbf\def\PY@tc##1{\textcolor[rgb]{0.00,0.00,1.00}{##1}}}
\expandafter\def\csname PY@tok@ne\endcsname{\let\PY@bf=\textbf\def\PY@tc##1{\textcolor[rgb]{0.82,0.25,0.23}{##1}}}
\expandafter\def\csname PY@tok@nv\endcsname{\def\PY@tc##1{\textcolor[rgb]{0.10,0.09,0.49}{##1}}}
\expandafter\def\csname PY@tok@no\endcsname{\def\PY@tc##1{\textcolor[rgb]{0.53,0.00,0.00}{##1}}}
\expandafter\def\csname PY@tok@nl\endcsname{\def\PY@tc##1{\textcolor[rgb]{0.63,0.63,0.00}{##1}}}
\expandafter\def\csname PY@tok@ni\endcsname{\let\PY@bf=\textbf\def\PY@tc##1{\textcolor[rgb]{0.60,0.60,0.60}{##1}}}
\expandafter\def\csname PY@tok@na\endcsname{\def\PY@tc##1{\textcolor[rgb]{0.49,0.56,0.16}{##1}}}
\expandafter\def\csname PY@tok@nt\endcsname{\let\PY@bf=\textbf\def\PY@tc##1{\textcolor[rgb]{0.00,0.50,0.00}{##1}}}
\expandafter\def\csname PY@tok@nd\endcsname{\def\PY@tc##1{\textcolor[rgb]{0.67,0.13,1.00}{##1}}}
\expandafter\def\csname PY@tok@s\endcsname{\def\PY@tc##1{\textcolor[rgb]{0.73,0.13,0.13}{##1}}}
\expandafter\def\csname PY@tok@sd\endcsname{\let\PY@it=\textit\def\PY@tc##1{\textcolor[rgb]{0.73,0.13,0.13}{##1}}}
\expandafter\def\csname PY@tok@si\endcsname{\let\PY@bf=\textbf\def\PY@tc##1{\textcolor[rgb]{0.73,0.40,0.53}{##1}}}
\expandafter\def\csname PY@tok@se\endcsname{\let\PY@bf=\textbf\def\PY@tc##1{\textcolor[rgb]{0.73,0.40,0.13}{##1}}}
\expandafter\def\csname PY@tok@sr\endcsname{\def\PY@tc##1{\textcolor[rgb]{0.73,0.40,0.53}{##1}}}
\expandafter\def\csname PY@tok@ss\endcsname{\def\PY@tc##1{\textcolor[rgb]{0.10,0.09,0.49}{##1}}}
\expandafter\def\csname PY@tok@sx\endcsname{\def\PY@tc##1{\textcolor[rgb]{0.00,0.50,0.00}{##1}}}
\expandafter\def\csname PY@tok@m\endcsname{\def\PY@tc##1{\textcolor[rgb]{0.40,0.40,0.40}{##1}}}
\expandafter\def\csname PY@tok@gh\endcsname{\let\PY@bf=\textbf\def\PY@tc##1{\textcolor[rgb]{0.00,0.00,0.50}{##1}}}
\expandafter\def\csname PY@tok@gu\endcsname{\let\PY@bf=\textbf\def\PY@tc##1{\textcolor[rgb]{0.50,0.00,0.50}{##1}}}
\expandafter\def\csname PY@tok@gd\endcsname{\def\PY@tc##1{\textcolor[rgb]{0.63,0.00,0.00}{##1}}}
\expandafter\def\csname PY@tok@gi\endcsname{\def\PY@tc##1{\textcolor[rgb]{0.00,0.63,0.00}{##1}}}
\expandafter\def\csname PY@tok@gr\endcsname{\def\PY@tc##1{\textcolor[rgb]{1.00,0.00,0.00}{##1}}}
\expandafter\def\csname PY@tok@ge\endcsname{\let\PY@it=\textit}
\expandafter\def\csname PY@tok@gs\endcsname{\let\PY@bf=\textbf}
\expandafter\def\csname PY@tok@gp\endcsname{\let\PY@bf=\textbf\def\PY@tc##1{\textcolor[rgb]{0.00,0.00,0.50}{##1}}}
\expandafter\def\csname PY@tok@go\endcsname{\def\PY@tc##1{\textcolor[rgb]{0.53,0.53,0.53}{##1}}}
\expandafter\def\csname PY@tok@gt\endcsname{\def\PY@tc##1{\textcolor[rgb]{0.00,0.27,0.87}{##1}}}
\expandafter\def\csname PY@tok@err\endcsname{\def\PY@bc##1{\setlength{\fboxsep}{0pt}\fcolorbox[rgb]{1.00,0.00,0.00}{1,1,1}{\strut ##1}}}
\expandafter\def\csname PY@tok@kc\endcsname{\let\PY@bf=\textbf\def\PY@tc##1{\textcolor[rgb]{0.00,0.50,0.00}{##1}}}
\expandafter\def\csname PY@tok@kd\endcsname{\let\PY@bf=\textbf\def\PY@tc##1{\textcolor[rgb]{0.00,0.50,0.00}{##1}}}
\expandafter\def\csname PY@tok@kn\endcsname{\let\PY@bf=\textbf\def\PY@tc##1{\textcolor[rgb]{0.00,0.50,0.00}{##1}}}
\expandafter\def\csname PY@tok@kr\endcsname{\let\PY@bf=\textbf\def\PY@tc##1{\textcolor[rgb]{0.00,0.50,0.00}{##1}}}
\expandafter\def\csname PY@tok@bp\endcsname{\def\PY@tc##1{\textcolor[rgb]{0.00,0.50,0.00}{##1}}}
\expandafter\def\csname PY@tok@fm\endcsname{\def\PY@tc##1{\textcolor[rgb]{0.00,0.00,1.00}{##1}}}
\expandafter\def\csname PY@tok@vc\endcsname{\def\PY@tc##1{\textcolor[rgb]{0.10,0.09,0.49}{##1}}}
\expandafter\def\csname PY@tok@vg\endcsname{\def\PY@tc##1{\textcolor[rgb]{0.10,0.09,0.49}{##1}}}
\expandafter\def\csname PY@tok@vi\endcsname{\def\PY@tc##1{\textcolor[rgb]{0.10,0.09,0.49}{##1}}}
\expandafter\def\csname PY@tok@vm\endcsname{\def\PY@tc##1{\textcolor[rgb]{0.10,0.09,0.49}{##1}}}
\expandafter\def\csname PY@tok@sa\endcsname{\def\PY@tc##1{\textcolor[rgb]{0.73,0.13,0.13}{##1}}}
\expandafter\def\csname PY@tok@sb\endcsname{\def\PY@tc##1{\textcolor[rgb]{0.73,0.13,0.13}{##1}}}
\expandafter\def\csname PY@tok@sc\endcsname{\def\PY@tc##1{\textcolor[rgb]{0.73,0.13,0.13}{##1}}}
\expandafter\def\csname PY@tok@dl\endcsname{\def\PY@tc##1{\textcolor[rgb]{0.73,0.13,0.13}{##1}}}
\expandafter\def\csname PY@tok@s2\endcsname{\def\PY@tc##1{\textcolor[rgb]{0.73,0.13,0.13}{##1}}}
\expandafter\def\csname PY@tok@sh\endcsname{\def\PY@tc##1{\textcolor[rgb]{0.73,0.13,0.13}{##1}}}
\expandafter\def\csname PY@tok@s1\endcsname{\def\PY@tc##1{\textcolor[rgb]{0.73,0.13,0.13}{##1}}}
\expandafter\def\csname PY@tok@mb\endcsname{\def\PY@tc##1{\textcolor[rgb]{0.40,0.40,0.40}{##1}}}
\expandafter\def\csname PY@tok@mf\endcsname{\def\PY@tc##1{\textcolor[rgb]{0.40,0.40,0.40}{##1}}}
\expandafter\def\csname PY@tok@mh\endcsname{\def\PY@tc##1{\textcolor[rgb]{0.40,0.40,0.40}{##1}}}
\expandafter\def\csname PY@tok@mi\endcsname{\def\PY@tc##1{\textcolor[rgb]{0.40,0.40,0.40}{##1}}}
\expandafter\def\csname PY@tok@il\endcsname{\def\PY@tc##1{\textcolor[rgb]{0.40,0.40,0.40}{##1}}}
\expandafter\def\csname PY@tok@mo\endcsname{\def\PY@tc##1{\textcolor[rgb]{0.40,0.40,0.40}{##1}}}
\expandafter\def\csname PY@tok@ch\endcsname{\let\PY@it=\textit\def\PY@tc##1{\textcolor[rgb]{0.25,0.50,0.50}{##1}}}
\expandafter\def\csname PY@tok@cm\endcsname{\let\PY@it=\textit\def\PY@tc##1{\textcolor[rgb]{0.25,0.50,0.50}{##1}}}
\expandafter\def\csname PY@tok@cpf\endcsname{\let\PY@it=\textit\def\PY@tc##1{\textcolor[rgb]{0.25,0.50,0.50}{##1}}}
\expandafter\def\csname PY@tok@c1\endcsname{\let\PY@it=\textit\def\PY@tc##1{\textcolor[rgb]{0.25,0.50,0.50}{##1}}}
\expandafter\def\csname PY@tok@cs\endcsname{\let\PY@it=\textit\def\PY@tc##1{\textcolor[rgb]{0.25,0.50,0.50}{##1}}}

\def\PYZbs{\char`\\}
\def\PYZus{\char`\_}
\def\PYZob{\char`\{}
\def\PYZcb{\char`\}}
\def\PYZca{\char`\^}
\def\PYZam{\char`\&}
\def\PYZlt{\char`\<}
\def\PYZgt{\char`\>}
\def\PYZsh{\char`\#}
\def\PYZpc{\char`\%}
\def\PYZdl{\char`\$}
\def\PYZhy{\char`\-}
\def\PYZsq{\char`\'}
\def\PYZdq{\char`\"}
\def\PYZti{\char`\~}
% for compatibility with earlier versions
\def\PYZat{@}
\def\PYZlb{[}
\def\PYZrb{]}
\makeatother


    % For linebreaks inside Verbatim environment from package fancyvrb. 
    \makeatletter
        \newbox\Wrappedcontinuationbox 
        \newbox\Wrappedvisiblespacebox 
        \newcommand*\Wrappedvisiblespace {\textcolor{red}{\textvisiblespace}} 
        \newcommand*\Wrappedcontinuationsymbol {\textcolor{red}{\llap{\tiny$\m@th\hookrightarrow$}}} 
        \newcommand*\Wrappedcontinuationindent {3ex } 
        \newcommand*\Wrappedafterbreak {\kern\Wrappedcontinuationindent\copy\Wrappedcontinuationbox} 
        % Take advantage of the already applied Pygments mark-up to insert 
        % potential linebreaks for TeX processing. 
        %        {, <, #, %, $, ' and ": go to next line. 
        %        _, }, ^, &, >, - and ~: stay at end of broken line. 
        % Use of \textquotesingle for straight quote. 
        \newcommand*\Wrappedbreaksatspecials {% 
            \def\PYGZus{\discretionary{\char`\_}{\Wrappedafterbreak}{\char`\_}}% 
            \def\PYGZob{\discretionary{}{\Wrappedafterbreak\char`\{}{\char`\{}}% 
            \def\PYGZcb{\discretionary{\char`\}}{\Wrappedafterbreak}{\char`\}}}% 
            \def\PYGZca{\discretionary{\char`\^}{\Wrappedafterbreak}{\char`\^}}% 
            \def\PYGZam{\discretionary{\char`\&}{\Wrappedafterbreak}{\char`\&}}% 
            \def\PYGZlt{\discretionary{}{\Wrappedafterbreak\char`\<}{\char`\<}}% 
            \def\PYGZgt{\discretionary{\char`\>}{\Wrappedafterbreak}{\char`\>}}% 
            \def\PYGZsh{\discretionary{}{\Wrappedafterbreak\char`\#}{\char`\#}}% 
            \def\PYGZpc{\discretionary{}{\Wrappedafterbreak\char`\%}{\char`\%}}% 
            \def\PYGZdl{\discretionary{}{\Wrappedafterbreak\char`\$}{\char`\$}}% 
            \def\PYGZhy{\discretionary{\char`\-}{\Wrappedafterbreak}{\char`\-}}% 
            \def\PYGZsq{\discretionary{}{\Wrappedafterbreak\textquotesingle}{\textquotesingle}}% 
            \def\PYGZdq{\discretionary{}{\Wrappedafterbreak\char`\"}{\char`\"}}% 
            \def\PYGZti{\discretionary{\char`\~}{\Wrappedafterbreak}{\char`\~}}% 
        } 
        % Some characters . , ; ? ! / are not pygmentized. 
        % This macro makes them "active" and they will insert potential linebreaks 
        \newcommand*\Wrappedbreaksatpunct {% 
            \lccode`\~`\.\lowercase{\def~}{\discretionary{\hbox{\char`\.}}{\Wrappedafterbreak}{\hbox{\char`\.}}}% 
            \lccode`\~`\,\lowercase{\def~}{\discretionary{\hbox{\char`\,}}{\Wrappedafterbreak}{\hbox{\char`\,}}}% 
            \lccode`\~`\;\lowercase{\def~}{\discretionary{\hbox{\char`\;}}{\Wrappedafterbreak}{\hbox{\char`\;}}}% 
            \lccode`\~`\:\lowercase{\def~}{\discretionary{\hbox{\char`\:}}{\Wrappedafterbreak}{\hbox{\char`\:}}}% 
            \lccode`\~`\?\lowercase{\def~}{\discretionary{\hbox{\char`\?}}{\Wrappedafterbreak}{\hbox{\char`\?}}}% 
            \lccode`\~`\!\lowercase{\def~}{\discretionary{\hbox{\char`\!}}{\Wrappedafterbreak}{\hbox{\char`\!}}}% 
            \lccode`\~`\/\lowercase{\def~}{\discretionary{\hbox{\char`\/}}{\Wrappedafterbreak}{\hbox{\char`\/}}}% 
            \catcode`\.\active
            \catcode`\,\active 
            \catcode`\;\active
            \catcode`\:\active
            \catcode`\?\active
            \catcode`\!\active
            \catcode`\/\active 
            \lccode`\~`\~ 	
        }
    \makeatother

    \let\OriginalVerbatim=\Verbatim
    \makeatletter
    \renewcommand{\Verbatim}[1][1]{%
        %\parskip\z@skip
        \sbox\Wrappedcontinuationbox {\Wrappedcontinuationsymbol}%
        \sbox\Wrappedvisiblespacebox {\FV@SetupFont\Wrappedvisiblespace}%
        \def\FancyVerbFormatLine ##1{\hsize\linewidth
            \vtop{\raggedright\hyphenpenalty\z@\exhyphenpenalty\z@
                \doublehyphendemerits\z@\finalhyphendemerits\z@
                \strut ##1\strut}%
        }%
        % If the linebreak is at a space, the latter will be displayed as visible
        % space at end of first line, and a continuation symbol starts next line.
        % Stretch/shrink are however usually zero for typewriter font.
        \def\FV@Space {%
            \nobreak\hskip\z@ plus\fontdimen3\font minus\fontdimen4\font
            \discretionary{\copy\Wrappedvisiblespacebox}{\Wrappedafterbreak}
            {\kern\fontdimen2\font}%
        }%
        
        % Allow breaks at special characters using \PYG... macros.
        \Wrappedbreaksatspecials
        % Breaks at punctuation characters . , ; ? ! and / need catcode=\active 	
        \OriginalVerbatim[#1,codes*=\Wrappedbreaksatpunct]%
    }
    \makeatother

    % Exact colors from NB
    \definecolor{incolor}{HTML}{303F9F}
    \definecolor{outcolor}{HTML}{D84315}
    \definecolor{cellborder}{HTML}{CFCFCF}
    \definecolor{cellbackground}{HTML}{F7F7F7}
    
    % prompt
    \makeatletter
    \newcommand{\boxspacing}{\kern\kvtcb@left@rule\kern\kvtcb@boxsep}
    \makeatother
    \newcommand{\prompt}[4]{
        {\ttfamily\llap{{\color{#2}[#3]:\hspace{3pt}#4}}\vspace{-\baselineskip}}
    }
    

    
    % Prevent overflowing lines due to hard-to-break entities
    \sloppy 
    % Setup hyperref package
    \hypersetup{
      breaklinks=true,  % so long urls are correctly broken across lines
      colorlinks=true,
      urlcolor=urlcolor,
      linkcolor=linkcolor,
      citecolor=citecolor,
      }
    % Slightly bigger margins than the latex defaults
    
    \geometry{verbose,tmargin=1in,bmargin=1in,lmargin=1in,rmargin=1in}
    
    

\begin{document}
    
    \maketitle
    
    

    
    \hypertarget{mini-project-2021}{%
\section{Mini Project 2021}\label{mini-project-2021}}

            \begin{tcolorbox}[breakable, size=fbox, boxrule=.5pt, pad at break*=1mm, opacityfill=0]
\begin{Verbatim}[commandchars=\\\{\}]
<IPython.core.display.HTML object>
\end{Verbatim}
\end{tcolorbox}
        
    \hypertarget{q1-sql-database}{%
\subsection{Q1 SQL Database}\label{q1-sql-database}}

\hypertarget{connect-to-the-mysql-database-which-has-a-table-of-samples-of-the-historical-freddie-mac-singlefamily-loan-level-data.}{%
\paragraph{Connect to the MySQL database which has a table of samples of
the historical Freddie Mac singlefamily loan level
data.}\label{connect-to-the-mysql-database-which-has-a-table-of-samples-of-the-historical-freddie-mac-singlefamily-loan-level-data.}}

    The following shows the connection details

    \begin{Verbatim}[commandchars=\\\{\}]
<mysql.connector.connection.MySQLConnection object at 0x000001E07F98E2E0>
    \end{Verbatim}

    The following is a list of databases fetched from MySQL

    \begin{Verbatim}[commandchars=\\\{\}]
[('information\_schema',), ('AgencyData',), ('mysql',), ('performance\_schema',)]
    \end{Verbatim}

    As we've connected with the database, we can now get the required sample
data table

            \begin{tcolorbox}[breakable, size=fbox, boxrule=.5pt, pad at break*=1mm, opacityfill=0]
\begin{Verbatim}[commandchars=\\\{\}]
  first\_pmt\_date    age  status first\_time\_ho\_flag  msa\_code mi\_pct  \textbackslash{}
0         200601   93.0  Prepay                  N   45460.0    000
1         200707  117.0   Alive                  N       NaN     12
2         199910   28.0  Prepay                  Y   12420.0     20
3         200705  119.0   Alive                  N   39900.0    000
4         201504   24.0   Alive                      40140.0    000

   credit\_score  num\_units occupancy\_status orig\_cltv orig\_dti orig\_upb  \textbackslash{}
0         635.0          1                O        76       55    57000
1         624.0          1                O        85       52   202000
2         680.0          1                O        75       56   237000
3         724.0          1                O        49       39   110000
4         794.0          1                O        49       40   150000

  orig\_ir prop\_type state    zip  loan\_seq\_num loan\_purpose  num\_borrowers
0   6.125        SF    IN  47800  F105Q4357152            N            2.0
1    6.25        SF    WI  53100  F107Q2066413            C            2.0
2    7.75        PU    TX  78700  F199Q3132223            P            1.0
3   5.625        SF    NV  89500  F107Q1290916            C            2.0
4    3.75        SF    CA  92500  F115Q1229673            N            2.0
\end{Verbatim}
\end{tcolorbox}
        
    \hypertarget{q2-analyse-mortgage-statuses}{%
\subsection{Q2 Analyse Mortgage
Statuses}\label{q2-analyse-mortgage-statuses}}

\hypertarget{analyze-the-breakdown-of-mortgage-statuses-prepay-default-alive-for-each-origination-cohort-group-by-year-of-first_pmt_date.}{%
\paragraph{Analyze the breakdown of mortgage statuses (prepay, default,
alive) for each origination cohort (group by year of
first\_pmt\_date).}\label{analyze-the-breakdown-of-mortgage-statuses-prepay-default-alive-for-each-origination-cohort-group-by-year-of-first_pmt_date.}}

    We first extract the year from first\_pmt\_date, then create a pivot
table, grouping on the index `year' and columns as different values of
`status'

    \hypertarget{graph-the-count-of-mortgages-in-each-status-stacked-y-axis-by-origination-cohort-x-axis.}{%
\paragraph{Graph the count of mortgages in each status (stacked y-axis)
by origination cohort
(x-axis).}\label{graph-the-count-of-mortgages-in-each-status-stacked-y-axis-by-origination-cohort-x-axis.}}

            \begin{tcolorbox}[breakable, size=fbox, boxrule=.5pt, pad at break*=1mm, opacityfill=0]
\begin{Verbatim}[commandchars=\\\{\}]
<AxesSubplot:xlabel='year'>
\end{Verbatim}
\end{tcolorbox}
        
    \begin{center}
    \adjustimage{max size={0.9\linewidth}{0.9\paperheight}}{Unison_Markdown_files/Unison_Markdown_15_1.png}
    \end{center}
    { \hspace*{\fill} \\}
    
    \hypertarget{comment-on-the-change-in-mortgage-behavior-through-the-mortgage-origination-cycle.}{%
\paragraph{Comment on the change in mortgage behavior through the
mortgage origination
cycle.}\label{comment-on-the-change-in-mortgage-behavior-through-the-mortgage-origination-cycle.}}

    We are likely to see more and more prepayments (or defaults) as the
mortgages age, as consumers may see value in switching (or default if
they're unable to pay the installments), as the economic conditions
change. Thus we see very few old loans alive, while most of the recent
ones are still active.

    \hypertarget{q3-analyse-mortgage-statuses}{%
\subsection{Q3 Analyse Mortgage
Statuses}\label{q3-analyse-mortgage-statuses}}

\hypertarget{repeat-the-analysis-above-but-instead-of-grouping-the-mortgages-by-first_pmt_date-group-them-by-age.}{%
\paragraph{Repeat the analysis above but instead of grouping the
mortgages by first\_pmt\_date, group them by
age.}\label{repeat-the-analysis-above-but-instead-of-grouping-the-mortgages-by-first_pmt_date-group-them-by-age.}}

            \begin{tcolorbox}[breakable, size=fbox, boxrule=.5pt, pad at break*=1mm, opacityfill=0]
\begin{Verbatim}[commandchars=\\\{\}]
<AxesSubplot:xlabel='age'>
\end{Verbatim}
\end{tcolorbox}
        
    \begin{center}
    \adjustimage{max size={0.9\linewidth}{0.9\paperheight}}{Unison_Markdown_files/Unison_Markdown_19_1.png}
    \end{center}
    { \hspace*{\fill} \\}
    
    \hypertarget{in-addition-briefly-explain-how-you-may-go-about-building-prepayment-and-default-models-for-residential-mortgages.-what-data-will-be-required-if-we-want-to-build-such-models-for-unison-home-equity-agreements-uea}{%
\paragraph{In addition, briefly explain how you may go about building
prepayment and default models for residential mortgages. What data will
be required if we want to build such models for Unison home equity
agreements
(UEA)?}\label{in-addition-briefly-explain-how-you-may-go-about-building-prepayment-and-default-models-for-residential-mortgages.-what-data-will-be-required-if-we-want-to-build-such-models-for-unison-home-equity-agreements-uea}}

    To build a prepayment and/or default model, we need to consider which
factors play a part in changing the status of a mortgage from `Alive' to
either `Prepay' or `Default'. The same can be achieved through logistic
regression. We can also look at the CPR(t) - the conditional prepayment
rate - to build the model dependent on these factors such as interest
rate, fico score etc.

For Unison Home Equity Agreements (UEA), apart from the given factors,
we may also need the data related to * Home prices (for e.g.~HPI - Home
Price Index), as higher prices also act as an incentive for borrowers to
go for higher mortgage or selling the property. * Economic Indicators
like GDP growth rate, unemployment rate etc. A booming economy would
lead to higher turnover in residential homes, as owners may want to
expand or move to a better location/house.

    \hypertarget{q4-unconditional-probabilities-of-mortgage-behavior}{%
\subsection{Q4 Unconditional Probabilities of Mortgage
Behavior}\label{q4-unconditional-probabilities-of-mortgage-behavior}}

\hypertarget{determine-the-unconditional-probabilities-of-mortgage-behavior-based-on-the-sample-data-and-describe-how-you-arrive-at-your-results.}{%
\paragraph{Determine the unconditional probabilities of mortgage
behavior based on the sample data and describe how you arrive at your
results.}\label{determine-the-unconditional-probabilities-of-mortgage-behavior-based-on-the-sample-data-and-describe-how-you-arrive-at-your-results.}}

    We are looking at the probability of a mortgage either defaulting,
prepaying or staying alive in 5 years. Since these are the only 3
possibilities, the total of these probabilities must be 1. To look for
unconditional probability, we need to make the following changes: *
Remove all loans which are alive and have age less than 5 years (60
months) - Since we do not know about the end state (at 5 years) of the
loan which is alive at age 2 years, we need to remove such loans from
the dataset. * Change status of all loans above 5 years to `Alive' - The
loans marked `Prepay' or `Default' after 5 years would all be alive at
the end of 5 years.

            \begin{tcolorbox}[breakable, size=fbox, boxrule=.5pt, pad at break*=1mm, opacityfill=0]
\begin{Verbatim}[commandchars=\\\{\}]
<AxesSubplot:ylabel='status'>
\end{Verbatim}
\end{tcolorbox}
        
    \begin{center}
    \adjustimage{max size={0.9\linewidth}{0.9\paperheight}}{Unison_Markdown_files/Unison_Markdown_24_1.png}
    \end{center}
    { \hspace*{\fill} \\}
    
    \hypertarget{how-could-prepayment-or-default-behavior-of-the-mortgage-affect-the-unison-equity-agreement-uea-on-the-same-home}{%
\paragraph{How could prepayment or default behavior of the mortgage
affect the Unison equity agreement (UEA) on the same
home?}\label{how-could-prepayment-or-default-behavior-of-the-mortgage-affect-the-unison-equity-agreement-uea-on-the-same-home}}

    A prepayment may be refinance due to lower rates or cash-out, which
imply a rise in house price. In case of turnover as well, the borrower
will also prepay and close the loan. All these scenarios are beneficial
for Unison as the Equity value in the house goes up, which stands to
benefit both Unison and UEA holder. Default on the other hand signifies
an economic hardship, which may result in a distress sale or auction of
property by lender. Such an event would negatively impact the value of
house, and the return for Unison in the UEA.

    \hypertarget{q5-prepayment-model}{%
\subsection{Q5 Prepayment Model}\label{q5-prepayment-model}}

\hypertarget{select-one-or-more-factors-which-you-believe-would-affect-the-prepay-response-and-build-a-simple-model-to-predict-prepayment-or-default-rates}{%
\paragraph{Select one or more factors which you believe would affect the
`Prepay' response and build a simple model to predict prepayment or
default
rates}\label{select-one-or-more-factors-which-you-believe-would-affect-the-prepay-response-and-build-a-simple-model-to-predict-prepayment-or-default-rates}}

    I'm building a model for Prepayment, in which case prepayment would
signify 1 while Alive or Default would be 0 in the logistic regression.
The dummy variables would also need to be created for categorical
columns. We can run an

    \begin{Verbatim}[commandchars=\\\{\}]
                            OLS Regression Results
==============================================================================
Dep. Variable:                      Y   R-squared:                       0.314
Model:                            OLS   Adj. R-squared:                  0.313
Method:                 Least Squares   F-statistic:                     377.0
Date:                Tue, 16 Feb 2021   Prob (F-statistic):               0.00
Time:                        23:58:20   Log-Likelihood:                -5051.6
No. Observations:               13188   AIC:                         1.014e+04
Df Residuals:                   13171   BIC:                         1.026e+04
Df Model:                          16
Covariance Type:            nonrobust
================================================================================
========
                           coef    std err          t      P>|t|      [0.025
0.975]
--------------------------------------------------------------------------------
--------
const                   -0.7381      0.064    -11.491      0.000      -0.864
-0.612
age                     -0.0031   8.13e-05    -38.664      0.000      -0.003
-0.003
credit\_score             0.0009   6.21e-05     13.947      0.000       0.001
0.001
orig\_cltv               -0.0016      0.000     -7.747      0.000      -0.002
-0.001
orig\_dti                -0.0003      0.000     -1.078      0.281      -0.001
0.000
orig\_ir                  0.1652      0.003     58.509      0.000       0.160
0.171
num\_borrowers            0.0612      0.006      9.545      0.000       0.049
0.074
first\_time\_ho\_flag\_Y    -0.0571      0.009     -6.069      0.000      -0.075
-0.039
occupancy\_status\_O       0.1210      0.018      6.709      0.000       0.086
0.156
occupancy\_status\_S       0.1943      0.025      7.757      0.000       0.145
0.243
prop\_type\_CP            -0.0059      0.057     -0.103      0.918      -0.118
0.107
prop\_type\_LH             0.0176      0.145      0.121      0.903      -0.268
0.303
prop\_type\_MH            -0.1392      0.040     -3.460      0.001      -0.218
-0.060
prop\_type\_PU             0.0170      0.014      1.238      0.216      -0.010
0.044
prop\_type\_SF             0.0139      0.012      1.154      0.248      -0.010
0.037
loan\_purpose\_N           0.0537      0.009      5.734      0.000       0.035
0.072
loan\_purpose\_P          -0.0805      0.009     -9.097      0.000      -0.098
-0.063
==============================================================================
Omnibus:                      869.205   Durbin-Watson:                   1.994
Prob(Omnibus):                  0.000   Jarque-Bera (JB):              750.724
Skew:                          -0.512   Prob(JB):                    9.60e-164
Kurtosis:                       2.435   Cond. No.                     3.49e+04
==============================================================================

Notes:
[1] Standard Errors assume that the covariance matrix of the errors is correctly
specified.
[2] The condition number is large, 3.49e+04. This might indicate that there are
strong multicollinearity or other numerical problems.
    \end{Verbatim}

    Looking at the t-stats for above variables, we can discard prop\_type
and consider everything else in the model. This just indicates that
property type does not significantly impact the decision to prepay a
loan. The most significant factors, from the above, are interest rates,
loan age and credit score, which are usually quoted as major factors for
borrower behaviour. We can now run a logistic regression, the score of
which will be displayed at the end. A high score denotes how well the
model is performing.

    \begin{Verbatim}[commandchars=\\\{\}]
Logistic Model Score:  0.83128601759175
    \end{Verbatim}

    The following conusion matrix shows the performance of the model vs the
actual prepayments.

    \begin{center}
    \adjustimage{max size={0.9\linewidth}{0.9\paperheight}}{Unison_Markdown_files/Unison_Markdown_38_0.png}
    \end{center}
    { \hspace*{\fill} \\}
    
    \hypertarget{comment-on-how-your-selected-response-prepayment-or-default-might-affect-the-market-price-of-a-mortgage-and-how-you-could-devise-an-investment-strategy-around-this-models-output.}{%
\paragraph{Comment on how your selected response (prepayment or default)
might affect the market price of a mortgage and how you could devise an
investment strategy around this model's
output.}\label{comment-on-how-your-selected-response-prepayment-or-default-might-affect-the-market-price-of-a-mortgage-and-how-you-could-devise-an-investment-strategy-around-this-models-output.}}

    By building the model, we can predict the cashflows of the mortgage.
These cashflows discounted back would then be used to arrive at the
value of the mortgage. If the model is good, we can identify the
mortgages (MBS/MSR) which are mispriced, earning an arbitrage in the
process.

    \hypertarget{q6-why-monitoring-is-essential-for-unison}{%
\subsection{Q6 Why monitoring is essential for
Unison}\label{q6-why-monitoring-is-essential-for-unison}}

\hypertarget{why-do-you-think-we-need-to-monitor-analyze-and-understand-the-mortgage-industry-in-general-and-our-customers-mortgages-in-particular}{%
\paragraph{Why do you think we need to monitor, analyze and understand
the mortgage industry in general and our customers' mortgages in
particular?}\label{why-do-you-think-we-need-to-monitor-analyze-and-understand-the-mortgage-industry-in-general-and-our-customers-mortgages-in-particular}}

    By buying borrower's equity, Unison is a party to the performance of
home equity. The mortgage may be a lead indicator, and the preformance
of mortgage and its behavior can tell us whether ur investment is doing
well or poorly, and allow us to manage our risk better. Also, a good
borrower is required for both mortgage as well as UEA, as a major risk
is tied to the economic condition of the borrower. Analyzing these
factors would improve our portfolio performance and mitigage the risk.

    \hypertarget{name-key-metrics-that-you-believe-are-good-indicators-of-risk-and-explain-how-you-think-we-can-use-those-metrics-in-risk-management-investment-structuring-etc.}{%
\paragraph{Name key metrics that you believe are good indicators of risk
and explain how you think we can use those metrics in risk management,
investment structuring,
etc.}\label{name-key-metrics-that-you-believe-are-good-indicators-of-risk-and-explain-how-you-think-we-can-use-those-metrics-in-risk-management-investment-structuring-etc.}}

    The major risk indicators are: 1. Housing prices - since we're buying
home equity, this becomes the primary risk for the portfolio. 2.
Economic conditions (GDP growth rate, unemployment rate,
inflation/interest rate) - these macro factors can help us in
determining whether the returns expected on the portfolio are good or
bad. 3. Borrower characteristics (fico score,


    % Add a bibliography block to the postdoc
    
    
    
\end{document}
